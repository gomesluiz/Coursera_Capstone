% Autogenerated translation of README.md by Texpad
% To stop this file being overwritten during the typeset process, please move or remove this header

\documentclass[12pt]{book}
\usepackage{graphicx}
\usepackage{fontspec}
\usepackage[utf8]{inputenc}
\usepackage[a4paper,left=.5in,right=.5in,top=.3in,bottom=0.3in]{geometry}
\setlength\parindent{0pt}
\setlength{\parskip}{\baselineskip}
\setmainfont{Helvetica Neue}
\usepackage{hyperref}
\pagestyle{plain}
\begin{document}

\chapter*{CAPSTONE PROJECT: FINAL REPORT}

\section*{LOST PETS LOCALISATION AND SEGMENTATION}

\subsection*{1. PURPOSE}

\hrule
\subsection*{2. INTRODUCTION}

\hrule
Missed and rejected pets have become a severe problem in many cities. Pets owners usually take a long time to find them or even they may never find their pets because, for example, these pets might be moved far from where they used to live. On the other hand,  ONG's have high difficulty to rescue rejected pets, so these ONGs are not often alerted about them or when are these pets have already moved to another place. 

The mobile phones have already become part of the day-to-day of people's lives, who might provide using their mobile phones the geographic location and a photo of these missed or rejected pets. In its turn, pets owners and ONGs could search and visualize the geographic location of missed or rejected pets including points of references such as cafes and restaurants. Furthermore, governmental agencies could visualize and compare areas of the city with the highest incidence of missed ou rejected and could segment areas to uncover patterns which could guide marketing campaigns to mitigate the problem.

\subsection*{3. DATA DESCRIPTION AND ACQUISITION}

\hrule
This prototype will make use of the following data sources:

\subsubsection*{Animal Services of The City of Toronto}

The \textbf{Stray Animals Report} provide by The Animal Services of The City of Toronto displays stray animals  (cats and dogs) received in the last 5 days. The report data will be scraped from https://www.toronto.ca/data/mls/animals/strayanimals.html, and contains the following information:

\begin{itemize}
\item \textbf{Category}
\item \textbf{Date} \\
\item \textbf{Breed} 
\item \textbf{Approximate Age} 
\item \textbf{Sex} 
\item \textbf{Colour}
\item \textbf{Receiving Shelter}
\item \textbf{Animal ID Number} 
\item \textbf{Photo}
\item \textbf{Crossing Intersection}
\end{itemize}

 The \textbf{Localisation of Receiving Shelters} provide by The Animal Services of City of Toronto contains the following information:

\begin{itemize}
\item \textbf{Title}
\item \textbf{Address} \\
\end{itemize}

Data will be scraped from https://www.toronto.ca/community-people/animals-pets/animal-shelters/.

\textbf{Toronto Venues nearby Crossing Intersections from FourSquare API (FourSquare website: www.foursquare.com)}

I will be using the FourSquare API to explore neighborhoods in selected towns in Singapore. The Foursquare explore function will be used to get the most common venue categories in each neighborhood, and then use this feature to group the neighborhoods into clusters. The following information are retrieved on the first query:

Venue ID
Venue Name
Coordinates : Latitude and Longitude
Category Name

Another venue query will be performed to retrieve venue ratings for each location. Note that rating information is a paid service from FourSquare and we are limited to only 50 queries per day. With this constraint, we limit the category analysis with only one type for this demo. I will try to retrieve as many ratings as possible for each retrieved venue ID.

\section*{4. METODOLOGY}

\section*{5. SEGMENTATION AND CLUSTERING}

\section*{6. DISCUSSION AND CONCLUSION}

\end{document}
